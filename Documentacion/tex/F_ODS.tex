\apendice{Anexo de sostenibilización curricular}

\section{Introducción}
Este apartado se incluye una reflexión personal sobre los aspectos de la sostenibilidad abordados en el presente Trabajo de Fin de Grado (TFG)  \textit{Análisis de imágenes hiperespectrales para la detección de cobre en viñedo (proyecto Dig4Vitis v2)}. Se detallan las competencias de sostenibilidad adquiridas y aplicadas durante el desarrollo del proyecto. Para ello se ha usado como guía y base sobre la que fundamentar dichas conclusiones las directrices de sostenibilidad de la CRUE \cite{Crue2012Sostenibilidad}
El objetivo principal de este apartado es indicar y desarrollar aquellos puntos en los que este trabajo contribuye a los principios de desarrollo sostenible, desde un prisma tanto tecnológico como social.

\section{Sostenibilidad Ambiental: Hacia una Agricultura más Eficiente}

El objetivo principal del proyecto realizado se condensa en la detección y cuantificación de tratamientos antifúngicos con base de cobre en hojas de viñedo. Dicho objetivo tiene un impacto directo en la sostenibilidad ambiental. En la actualidad la aplicación de estos tratamientos se realiza de manera indiscriminada resultando en un uso excesivo de los recursos, posible contaminación del suelo y el agua teniendo un impacto negativo sobre la biodiversidad.

Este proyecto se enmarca dentro del concepto de agricultura de precisión. Dicho concepto tiene como objetivo la optimización de los recursos agrícolas. En este caso al realizar una cuantificación precisa de la cantidad de producto en la superficie de la hoja, la herramienta desarrollada facilita una aplicación correcta y óptima del producto evitando desperdiciar recursos. Reduciendo y optimizando los recursos empleados se obtiene un impacto positivo directo sobre la huella ecológica del uso de estos productos antifúngicos en el entorno de la viticultura. Un menor uso de cobre implica:
\begin{itemize}
    \item \textbf{Reducción de la contaminación}: Menos químicos en el ambiente se traduce en una menor filtración y posterior contaminación hacia acuíferos y suelos adyacentes.
    \item \textbf{Conservación de recursos}: La producción y distribución de estos fungicidas requieren energía y materiales; su uso eficiente conserva estos recursos.
    \item \textbf{Impacto en la biodiversidad}: La reducción de químicos puede proteger a organismos no objetivo, como insectos polinizadores o microorganismos del suelo ayudando a no reducir la biodiversidad en la zona donde se aplica el producto.

\end{itemize}

La tecnología hiperespectral empleada para la recopilación de los datos y evaluación de los cultivos es en sí misma no invasiva y no destructiva. Por este motivo este tipo de tecnología es sostenible en comparación a con otros tipos de muestreo que implican destrucción de los cultivos o parte de ellos para la obtención de las muestras y datos.  La capacidad de analizar grandes áreas de forma eficiente sin alterar el ecosistema es un pilar fundamental de la sostenibilidad en la agricultura moderna.

\section{Sostenibilidad Social: Mejora de la salud pública y costes de cultivo}

Aunque menos directa, la contribución de este proyecto a la sostenibilidad social es igualmente relevante. Una agricultura más eficiente y sostenible beneficia a la sociedad en varios aspectos:
\begin{itemize}
    \item \textbf{Salud Pública}: La reducción en el uso de fungicidas disminuye la exposición de los trabajadores agrícolas a químicos potencialmente dañinos y reduce los residuos de productos en los alimentos finales, mejorando la calidad de vida de los trabajadores y evitando posibles riesgos para la salud de los consumidores finales del producto.

    \item \textbf{Sostenibilidad Económica para el Agricultor:} La optimización de los tratamientos conduce a una reducción de costes para los agricultores, al gastar menos en fungicidas. Esto mejora la viabilidad económica de las explotaciones agrarias, especialmente para pequeños y medianos viticultores.

\end{itemize}

La posibilidad de descargar resultados y analizarlos detalladamente  fomenta la transparencia y el seguimiento de las prácticas agrícolas. Esto es crucial para la confianza del consumidor y la adopción de estándares de calidad.

\section{Sostenibilidad Económica: Optimización y Rentabilidad Agrícola}

Como se ha mencionado con anterioridad el objetivo del proyecto es permitir a los agricultores hacer un uso más eficiente de sus recursos. Gracias a la detección precisa de las zonas tratadas con producto.
Causando así un impacto directo en aspectos como:
\begin{itemize}
    \item \textbf{Reducción de costes:} Menos fungicida significa menos gasto en el producto en sí.
    \item \textbf{Optimización de mano de obra y maquinaria:} Las aplicaciones más precisas pueden reducir el tiempo y el esfuerzo necesarios, así como el consumo de combustible de la maquinaria agrícola.
    \item \textbf{Mejora de la eficiencia:} Uno de los resultados que tiene el asegurar una cobertura adecuada solo donde es necesaria, es maximizar la efectividad del tratamiento, protegiendo mejor la cosecha y reduciendo las pérdidas por enfermedades.
\end{itemize}

\section{Contribución personal a la Sostenibilidad}

Desde una perspectiva personal, el desarrollo de este Trabajo de Fin de Grado me ha permitido no solo reflexionar sobre como un grado como el de Ingeniería Informática puede tener un impacto en la sostenibilidad, sino ver de primera mano cómo mediante herramientas como la que se ha desarrollado en este proyecto pueden ser de gran utilidad a la hora de abordar desafíos de sostenibilidad. Gracias e este proyecto he sido capaz de obtener una mejor comprensión práctica de como la tecnología se puede aplicar para lograr objetivos ambientales y económicos.
Aspectos como la eficiencia del código y la reusabilidad del mismo también se alinean con los principios de sostenibilidad, buscando soluciones que optimicen los recursos computacionales y faciliten la futura implementación de posibles mejoras, además de prolongar la vida útil del software. 
Este proyecto me ha proporcionado una oportunidad muy valiosa para poder integrar la concienciación ambiental y social en el desarrollo de software, estando en cierto modo tan alejados conceptos como la concienciación ambiental y social del desarrollo meramente técnico de una herramienta software, proporcionándome una competencia que considero ahora esencial para todos aquellos futuros ingenieros.

Este anexo busca subrayar que el proyecto Dig4Vitis v2 no es solo un avance tecnológico, sino una contribución tangible hacia una viticultura más sostenible, eficiente y respetuosa con el medio ambiente y la sociedad.
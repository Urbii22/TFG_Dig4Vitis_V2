\capitulo{2}{Objetivos del proyecto}

En este apartado se detallan los objetivos que se persiguen con la realización del proyecto. Se puede distinguir entre los objetivos marcados para los requisitos del software a construir y los objetivos de carácter técnico que se plantean a la hora de llevar a la práctica el proyecto.

\section {Objetivos Generales}
\begin{enumerate}

\item Desarrollar una aplicación, con una interfaz intuitiva, en la que el usuario pueda interactuar de forma sencilla y obtener un resultado visual y de forma inmediata de la eficacia de la aplicación de productos fungicidas, con base de cobre. La aplicación estará basada en el análisis de imágenes hiperespectrales de hojas de vid   

\item Validar una metodología de análisis de imagen hiperespectral que garantice la obtención de resultados coherentes, fiables y repetibles. El fin último es que la cuantificación del recubrimiento sea aplicable en entornos de investigación y con potencial para su uso en aplicaciones reales de agricultura de precisión. 

\end{enumerate}

\section {Objetivos Técnicos} 
Para alcanzar los objetivos generales, se establecieron las siguientes metas técnicas específicas, centradas en el diseño y la implementación de la herramienta:

\begin{enumerate}
    \item La construcción de una aplicación web interactiva en Python, utilizando la biblioteca Streamlit, que integre todo el flujo de trabajo: desde la carga de datos hiperespectrales hasta el procesamiento y la visualización final de los resultados.
    \item El diseño e implementación de un pipeline de visión por computador para el procesamiento de imágenes hiperespectrales. Este pipeline debe incluir módulos para la segmentación de la imagen (separación de fondo, hoja y producto) y para la gestión eficiente de los datos del hipercubo.
    \item La implementación de un sistema robusto para la eliminación de falsos positivos basado en la sustracción de ruido. Este objetivo es el núcleo técnico del proyecto e implica la alineación geométrica precisa de la imagen tratada con una imagen de control mediante el uso de los algoritmos ORB y RANSAC.
    \item La habilitación de funcionalidades para la exportación de los resultados, permitiendo al usuario descargar tanto las imágenes del análisis (las máscaras de segmentación y la imagen final con la reducción de ruido) como los datos cuantitativos del porcentaje de recubrimiento.
    \item El desarrollo de la solución siguiendo principios de ingeniería del software que aseguren una estructura de código clara, modular y documentada, facilitando así el mantenimiento, la reusabilidad de los componentes y la futura escalabilidad del proyecto.
    \item La creación de una interfaz de usuario (UI) y una experiencia de usuario (UX) que cumplan con criterios de usabilidad, haciendo que el manejo de una herramienta técnicamente compleja sea sencillo e intuitivo. 

\end{enumerate}

\section {Objetivos Personales} 
\begin{enumerate}
\item Aprender el ciclo de vida completo del desarrollo de una aplicación funcional, desde el concepto hasta el despliegue, utilizando el framework \verb|Streamlit|. 
\item  Mejorar en la comprensión del flujo de trabajo  e implementar buenas prácticas a la hora de realizar un desarrollo de un producto informático. Así como  mis habilidades de inclusión e implementación de patrones de diseño en el desarrollo de la aplicación, así como su impacto en el rendimiento y otros aspectos.
\item Aumentar y fortalecer mi comprensión, nivel de programación y comprensión del lenguaje Python así como adquirir experiencia práctica con las bibliotecas especializadas en visión por computador, además de la investigación y descubrimiento de nuevas librerías y algoritmos útiles para el desarrollo de la aplicación.
\item Mejorar a nivel personal en un entorno profesional con sus fechas límites, requisitos a cumplimentar, trabajo científico y buenas prácticas a la hora de desarrollar un producto.

\end{enumerate}

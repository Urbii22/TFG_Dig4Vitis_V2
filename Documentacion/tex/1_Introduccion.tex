\capitulo{1}{Introducción}

La aplicación desarrollada durante este Trabajo de fin de Grado analiza dos imágenes hiperespectrales de una misma hoja de viñedo a la que se le ha aplicado un tratamiento fungicida en espray. Las imágenes corresponden a la hoja sin el tratamiento aplicado e inmediatamente después de realizar el espray.
Las imágenes hiperespectrales empleadas en este trabajo contienen información para 300 bandas, desde los 380 nm hasta los 1000 nm. El empleo de estas imágenes en el análisis del dosel de hojas de viñedo ha demostrado ser un método efectivo y no destructivo de análisis en tiempo real \cite{Tosin2021, DiGennaro2022}.

Para el desarrollo de la aplicación se han desarrollado e integrado métodos avanzados para la alineación geométrica y superposición de imágenes, utilizando el algoritmo ORB para la detección de características morfológicas coincidentes entre las dos imágenes y RANSAC para una estimación robusta de la transformación entre ambas tomas. Así como el desarrollo de una herramienta para la eliminación del ruido espectral, con el fin de optimizar el resultado.

La aplicación es capaz de detectar las gotas de producto fungicida aplicado de forma similar al tratamiento real en viñedo y devolver una imagen de la doble binarización de la hoja de vid (trinarizada), en la que se puede distinguir la parte recubierta por el producto, además de indicar el porcentaje de recubrimiento con respecto a la superficie total de la hoja.




\section{Estructura de la memoria}
\begin{enumerate}
    \item \textbf{Introducción}: Descripción del proyecto y estructura de la documentación.
    \item \textbf{Objetivos del Proyecto}: Objetivos generales, técnicos y personales que se esperan cumplir en este proyecto.
    \item \textbf{Conceptos Teóricos}: Explicaciones teóricas sobre los conceptos más relevantes del proyecto.
    \item \textbf{Técnicas y Herramientas}: Explicación de las técnicas y herramientas que se han usado a lo largo del desarrollo del proyecto.
    \item \textbf{Aspectos Relevantes del Desarrollo del Proyecto}: Puntos interesantes e importantes que han ido surgiendo a lo largo del desarrollo del proyecto.
    \item \textbf{Trabajos Relacionados}: Trabajos relacionados con la temática tratada en este proyecto.
    \item \textbf{Conclusiones y Líneas de Trabajo Futuras}: Las conclusiones que se han extraído de la realización del proyecto y propuestas e ideas de por donde podría seguirse desarrollando este proyecto.
\end{enumerate}

\section{Estructura de los anexos}
     \begin{enumerate} [A.]
         \item \textbf{Plan de Proyecto Software}:
         \item \textbf{Especificación de Requisitos}:
         \item \textbf{Especificación de Diseño}:
         \item \textbf{Documentación técnica de programación}:
         \item \textbf{Documentación de Usuario}:
         \item \textbf{Anexo de Sostenibilidad Curricular}:
     \end{enumerate}



